%==================================================================================================
\thispagestyle{fancy}
\fancyhf{}
\lhead{\begin{picture}(225,120)%(horizontal,vertical)
{\includegraphics[scale=0.13]{pre/logo-unicamp.jpg}}
\end{picture}}
\chead{}
\rhead{}
\lfoot{}
\cfoot{\thepage}
\begin{center}
\vspace*{2.2cm}
{\large\textbf{UNIVERSIDADE ESTADUAL DE CAMPINAS\\\vspace{1.2ex}
FACULDADE DE ENGENHARIA MECÂNICA\vspace{1.2ex}}}

\vspace{1.5cm}
{\large Gilberto Luis Valente da Costa}

\vspace{1.3cm}
{\fontsize{23}{23} \textbf{$hp^2$FEM - Uma Arquitetura de \textit{Software}}}	\\ \vspace{1.2ex}
{\fontsize{23}{23} \textbf{$p$ Não-Uniforme para o Método de}}			\\ \vspace{1.2ex}
{\fontsize{23}{23} \textbf{Elementos Finitos de Alta Ordem}}
\end{center}

\vspace{0.7cm}
\noindent
Orientador: Prof. Dr. Marco Lúcio Bittencourt\\
\vspace{0.5cm}
% Coorientador:

\vspace{0.1cm}
Dissertação de Mestrado apresentada à Faculdade de Engenharia Mecânica da Universidade Estadual de
Campinas, para a obtenção do título de Mestre em Engenharia Mecânica, na Área de Mecânica dos
Sólidos e Projeto Mecânico.

\vspace{0.4cm}
\begin{flushleft}
\parbox{4in}{ESTE EXEMPLAR CORRESPONDE À VERSÃO FINAL DA DISSERTAÇÃO DEFENDIDA PELO ALUNO
Gilberto Luis Valente da Costa, E
ORIENTADO PELO PROF. DR.
Marco Lúcio Bittencourt.\\
\begin{center}
...............................................................\\
ASSINATURA DO ORIENTADOR
\end{center}
}
\end{flushleft}

\vspace{0.4cm}
\begin{center}
CAMPINAS\\ 2012
\end{center}

%==================================================================================================
% \thispagestyle{fancy}
% \fancyhf{}
% \lhead{\begin{picture}(225,120)%(horizontal,vertical)
% \put(0,0){\includegraphics[scale=0.13]{pre/logo-unicamp.jpg}}
% \end{picture}}
% \chead{}
% \rhead{}
% \cfoot{\thepage}
% \begin{center}
% \vspace*{2.2cm}
% {\large\textbf{UNIVERSIDADE ESTADUAL DE CAMPINAS\\\vspace{1.2ex}
% FACULDADE DE ENGENHARIA MECÂNICA\vspace{1.2ex}}}
% 
% \vspace{1.5cm}
% {\large Vinícius Augusto Diniz Silva}
% 
% \vspace{1.3cm}
% {\fontsize{23}{23} \textbf{Sobre a preparação de teses e dissertações\\ \vspace{1.2ex}
%  na Faculdade de Engenharia Mecânica}}
% \end{center}
% 
% \vspace{0.7cm}
% \noindent
% Orientador: \\
% \vspace{0.5cm}
% Coorientador:
% 
% \vspace{0.1cm}
% Dissertação de Mestrado apresentada à Faculdade de Engenharia Mecânica da Universidade Estadual de
% Campinas, para a obtenção do título de Mestre(a) em Engenharia Mecânica, na Área de
% ........................................
% 
% \vspace{0.4cm}
% \begin{flushleft}
% \parbox{4in}{ESTE EXEMPLAR CORRESPONDE À VERSÃO FINAL DA DISSERTAÇÃO DEFENDIDA PELO(A) ALUNA(A)
% ........................................................................................, E
% ORIENTADO(A) PELO(A) PROF(A). DR(A)
% ........................................................................................\\
% \begin{center}
% ...............................................................\\
% ASSINATURA DO(A) ORIENTADOR(A)
% \end{center}
% }
% \end{flushleft}
% 
% \vspace{0.4cm}
% \begin{center}
% CAMPINAS\\ 2012
% \end{center}
