%===================================================================================================
%%% Força a contagem da página.
\clearpage
\setcounter{page}{21}
%%%

\vspace*{1.cm}
\begin{center}
\chapter*{Lista de Abreviaturas e Siglas}
\end{center}

\vspace{1cm}
\noindent

\textbf{\emph{Matrizes e Vetores}}\\

\noindent
\begin{tabular}[!thp]{l c p{.72\linewidth} c}
$\left\{F \right\}$,$\left\{f \right\}$& - & Vetor de forças aplicadas\\
					& - & Termos independentes\\
$\left\{F_{e} \right\}$ & - & Vetor local de forças aplicadas\\
$F_{e,i}$ & - & Coeficientes do vetor local de forças aplicadas\\
$f^{1D}_{e}$ & - & Vetor de carga unidimensional associado à matriz de massa\\
$f^{1D}_{i}$ & - & i-ésimo elemento do vetor de carga unidimensional associado à matriz de massa\\
% $f^{1D,k}_{i}$ & - & i-ésimo elemento do vetor de carga unidimensional associado à matriz de
% rigidez\\
$\left\{ f^{m}_{1D} \right\}$ & - & Vetor de carga associado à matriz de massa unidimensional\\
% $\left\{ f^{k}_{1D} \right\}$ & - & Vetor de carga associado à matriz de rigidez unidimensional\\

$\left[K \right]$ & - & Matriz de rigidez global\\
% $\left[ K_{1D} \right]$ & - & Matriz de rigidez unidimensional\\
$\left[K_{e} \right]$ & - & Matriz de rigidez local\\
% $K^{1D}_{ij}$ & - & Coeficientes de matriz de rigidez unidimensional\\
% $K^{2D}_{ij}$ & - & Coeficientes de matriz de rigidez bidimensional\\

$\left[M \right]$ & - & Matriz de massa global\\
$\left[M_{i,j} \right]$ & - & Coeficientes da matriz de massa\\
$\left[ M_{1D} \right], \left[ M_{e}^{1D} \right]$ & - & Matriz de massa unidimensional\\
$\left[M_{e} \right]$ & - & Matriz de massa local\\
$\left[M_{e} \right]^{-1}$ & - & Inversa da matriz de massa local\\
$M_{e,ij}$ & - & Coeficientes da matriz de massa local\\
$M^{1D}_{e,ij}$ & - & Coeficientes da matriz de massa unidimensional\\
$M^{2D}_{ij}$ & - & Coeficientes da matriz de massa bidimensional\\

$\left[N_{Sol,Pos} \right]$ & - & Matriz de funções de interpolação da malha de solução nos pontos
de colocação da malha de pós-processamento\\

$\left\{q \right\}, \left\{u_{e}\right\}, \left\{a_{e}\right\}$ & - & Vetor de deslocamentos
locais\\

$\left\{R \right\}$ & - & Resíduo\\

$\left\{w \right\}$ & - & Função ponderadora\\

$\left\{u \right\}$,$\left\{a \right\}$ & - & Vetor de deslocamentos\\

\end{tabular}
%===================================================================================================


%===================================================================================================
%%% Força a contagem da página.
\clearpage
\setcounter{page}{22}
%%%

\noindent
\begin{tabular}[!thp]{l c p{.72\linewidth} c}
$\left\{u^{Pos}_{e} \right\}$ & - & Vetor de solução local da malha de pós-processamento\\
$\left\{u^{Sol}_{e} \right\}$ & - & Vetor de solução local da malha de solução\\
$\left\{ u^{1}_{e} \right\}, \left\{ u^{2}_{e} \right\}, \left\{ u^{3}_{e} \right\}$ & - & Vetores
de deslocamentos para os elementos 1, 2 e 3\\
\end{tabular}
\newline \newline


\textbf{\emph{Letras Latinas}}\\

\noindent
\begin{tabular}{l c p{.6\linewidth} c}
$b$ & - & Intensidade do termo independente\\
$b_{j}$ & - & Coeficientes de aproximação\\
$d, {}'$ & - & Derivada de primeira ordem\\
${}''$ & - & Derivada de segunda ordem\\
$e$ & - & Erro na aproximação polinomial\\
$\left | J \right |$ & - & Determinante do Jacobiano\\
$L^{p}$ & - & Norma $p$\\
$L_{j}$ & - & Comprimento de um elemento $j$\\
$L_{1}, L_{2}$ & - & Comprimento dos elementos 1 e 2\\
$R$ & - & Resíduo\\$w$ & - & Função ponderadora\\
$u(.)$ & - & Função polinomial qualquer\\
$u_{ap}(.)$ & - & Solução aproximada\\
$u_{i}$ & - & Deslocamentos nodais para o nó $i$\\
$u_{ij}$ & - & Deslocamentos nodais para o nó $i$ no elemento $j$\\
$u_{12}, u_{22}$ & - & Deslocamentos nodais do nó 2 nos elementos 1 e 2\\
$u_{11}, u_{12}, u_{22}, u_{23}, u_{33}, u_{34}$ & - & Deslocamentos nodais calculados nos
elementos 1, 2 e 3\\
\end{tabular}
\newline \newline


\textbf{\emph{Letras Gregas}}\\

\noindent
\begin{tabular}{l c p{.6\linewidth} l}
$\Omega$ & - & Domínio\\
$\xi_{i}$ & - & Coordenada local em [-1,1]\\
$\xi, \xi_{1}, \xi_{2}$ & - & Coordenadas locais\\
$\phi_{i},\phi_{j}$ & - & Funções de interpolação\\
$\left\{ \phi_{i} \right\}$ & - & Base formada por um conjunto de funções $\phi_{i}$\\
$\phi_{a}, \phi_{b}, \phi_{p}, \phi_{q} $& - & Funções de interpolação unidimensionais\\
$\phi_{1}, \phi_{2}, \phi_{3}	$& - & Funções de interpolação do polinômio de Lagrange\\
\end{tabular}
%===================================================================================================


%===================================================================================================
%%% Força a contagem da página.
\clearpage
\setcounter{page}{23}
%%%

\textbf{\emph{Siglas}}\\

\noindent
\begin{tabular}{l c p{.8\linewidth} c}
$\textbf{ACDP}$& - &Ambiente Computacional para Desenvolvimento de Programas\\
$\textbf{DOFs}$ & - & \textit{Degrees of Freedom} (Graus de Liberdade) \\
$\textbf{EDP}$ & - & Equações Diferenciais Parciais \\
$\textbf{GLC}$ & - & Gramática Livre de Contexto\\
$\textbf{$hp^2$FEM}$ & - & 
Programa do M\'{e}todo de Elementos Finitos de Alta performance - $2^{a}$ vers\~{a}o
(\emph{High Performace Finite Element Method Software})\\
$\textbf{MEF}$ & - & Método de Elementos Finitos\\
$\textbf{MEF-AO}$ & - & Método de Elementos Finitos de Alta Ordem\\
$\textbf{POO}$ & - & Programação Orientada a Objeto\\
$ $& - & Paradigma de Orientação a Objetos\\
$\textbf{UML}$ & - & \textit{Unified Modeling Language}\\
$\textbf{XMI}$ & - & \textit{XML Metadata Interchange}\\
$\textbf{XML}$ & - & \textit{eXtended Markup Language}\\
\end{tabular}
\newline \newline


\textbf{\emph{Outras Notações}}\\

\noindent
\begin{tabular}{l c p{.8\linewidth} l}
$p$-uniforme & - & Distribuição polinomial uniforme na malha de elementos finitos.\\
$p$-não-uniforme & - & Distribuição polinomial não uniforme na malha de elementos finitos.\\
$<,>$ & - & Produto interno entre dois vetores\\
$\left\{  \right\}^{T}$ & - & Transposto de um vetor\\
1D & - & Elemento unidimensional\\
2D & - & Elemento bidimensional\\
3D & - & Elemento tridimensional\\
\end{tabular}
%===================================================================================================
