\pagestyle{empty}

% Inclui o cabeçalho definido no meta.tex
\pagestyle{fancy}


%===================================================================================================
%                                              Capítulo 2 
%===================================================================================================

\chapter{Título do Capítulo 2}\label{cap2}
Nesse capítulo, apresentam-se conceitos sobre o método de elementos finitos ...


%---------------------------------------------------------------------------------------------------
\section{Título da Seção 1}\label{cap2:secao1}

Descrição . . .

Para utilizar as referências em latex acrescente o comando \begin{verbatim} ~\cite{referencia} 
\end{verbatim}.

Esses são 2 exemplos de citações: ~\cite{rylo2007} e~\cite{lima2008}.

O comando \begin{verbatim} ~\citet{referencia} \end{verbatim}

é utilizado quando se coloca o autor no começo da oração. Exemplo: 

~\citet{bittencourt2010} propôs o método X para o desenvolvimento de . . .
%---------------------------------------------------------------------------------------------------

%---------------------------------------------------------------------------------------------------
\subsection{Título da Subseção 1}\label{cap2:sub1}

Descrição . . .

Exemplo para definir imagens. Veja o arquivo capitulo2.tex e verifique como a 
Figura~\ref{fig:logo} é definida.

%   Exemplo de Figura.
\begin{figure}[!thp]
  \centering
  \includegraphics[width=4.cm, height=5.5cm]{./cap2/figs/logo-unicamp.jpg}
  \caption{Logo da Unicamp.}
  \label{fig:logo}
\end{figure}
%---------------------------------------------------------------------------------------------------

%---------------------------------------------------------------------------------------------------
\section{Título da Seção 2}\label{cap2:secao2}

Descrição . . .

Agora, um exemplo da para descrever as equações em latex. Veja o arquivo capitulo2.tex verifique 
como as equações~\ref{eq:massa1D} e~\ref{eq:carga1D} estão definidas.

\begin{equation}\label{eq:massa1D}
M^{1D}_{e,ij} = \int_{-1}^{1}\phi_{i}\phi_{j}(\xi_{1}) d\xi_{1},
\end{equation}
% E os carregamento de corpos para o problema de projeção \cite{santos2011} são
\begin{equation}\label{eq:carga1D}
f^{1D}_{i} = \int_{-1}^{1} b(\xi_{1})\phi_{i}(\xi_{1}) d\xi_{1},
\end{equation}

%---------------------------------------------------------------------------------------------------

%===================================================================================================

