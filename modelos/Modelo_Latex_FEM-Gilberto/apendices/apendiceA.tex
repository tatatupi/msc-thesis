\begin{center}
\appendix{-- \ \ Arquivos do analisador simbólico-numérico}\label{ApendiceA}
\end{center}



\section{Arquivo do analisador léxico}\label{ApendiceA:Lex}

\lstinputlisting[language=Perl, label=sources:AnalisadorLexico, caption={Arquivo para geração de
um analisador léxico}]{./apendices/sources/calculator.l}


\section{Arquivo do analisador sintático}\label{ApendiceA:Sintatico}

\lstinputlisting[language=Perl, label=sources:AnalisadorSintatico, caption={Arquivo para geração de
um analisador sintático}]{./apendices/sources/calculator.y}


\section{Arquivo de interface entre o $hp^2$FEM e o analisador
simbólico-numérico}\label{ApendiceA:Analisador}

\lstinputlisting[language=Perl, label=sources:AnalisadorSintatico, caption={Arquivo de interface
entre o \textit{software} $hp^2$FEM e o analisador
simbólico-numérico}]{./apendices/sources/calculator.y}



% Apêndices\index{apêndices} complementam o texto principal da tese com informações para leitores com
% especial interesse no tema, devendo ser considerados leitura opcional, ou seja, o entendimento do
% texto principal da tese não deve exigir a leitura atenta dos apêndices.
% 
% Apêndices usualmente contemplam provas de teoremas, deduções de fórmulas\index{fórmulas}
% matemáticas\index{fórmulas!matemáticas}, diagramas esquemáticos, gráficos e trechos de código.
% Quanto a este último, código extenso não deve fazer parte da tese, mesmo como apêndice. O ideal é
% disponibilizar o código na Internet para os interessados em examiná-lo ou utilizá-lo.
% 
% \section{Seção do apêndice}
% Equação:
% \begin{equation}\label{eq:WTS}
% w_i=X_{1}^i(x_1)t \ldots tX_{n}^i(x_n)
% \end{equation}
