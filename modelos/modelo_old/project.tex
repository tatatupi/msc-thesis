
%-------------------------------------------------------------------------
% Comentário adicional do PPgSI - Informações sobre ``figura''
% 
% Caption(Título) de tabelas e ilustração (tais como figura, gráfico, 
% algoritmo, fotografia, quadro etc.) sempre acima da própria.
%
% Para todos os captions/(títulos) (de seções, subseções, tabelas, 
% ilustrações, etc.):
%     - em maiúscula apenas a primeira letra da sentença (do título), 
%       exceto nomes próprios, geográficos, institucionais ou Programas ou
%       Projetos ou siglas, os quais podem ter letras em maiúscula também.
%
% Fonte de ilustração (tais como figura, gráfico, algoritmo, fotografia, 
% quadro etc.) sempre abaixo da própria.
%      - se a fonte for o próprio autor, colocar o nome dele. 
%      - se a fonte for outro autor, citar sua referência.
%
% Todas  as tabelas, ilustrações (figuras, quadros, gráficos etc. ), 
% anexos, apêndices devem obrigatoriamente ser citados no texto.
%      - a citação deve vir sempre antes da primeira vez em que a tabela, 
%        ilustração etc., aparecer pela primeira vez.
%
%-------------------------------------------------------------------------
%\begin{figure}[H]% H manda colocar exatamente nessa posição no texto (relativa aos parágrafos anterior e posterior)
%	\centering
% 	  \caption{Exemplo de título de ilustração do tipo figura, incluindo como ``Fonte:'' o próprio autor}
%		\includegraphics{figura-exemplo.png}
%	\label{fig:figura-exemplo1}
%  \source{Marcelo Fantinato, 2015}
%\end{figure}


%\begin{table}[htbp]
%	\centering
%	\caption{Exemplo de título de tabela}
%		\begin{tabular}{p{1in} p{1in} p{1in} p{1in} } \hline
%
%		Cabeçalho 1	& Cabeçalho 2	& Cabeçalho 3	& Cabeçalho 4 \\ \hline
%		Texto	& número & número	& número \\ 
%		Texto	& número & número	& número \\ 
%		Texto	& número & número	& número \\ 
%		Texto	& número & número	& número \\ 
%		Texto	& número & número	& número \\ \hline
%		
%		\end{tabular}
%	\label{tab:ExemploDeTabela1}
%  \source{Marcelo Fantinato, 2015}
%\end{table}


%-------------------------------------------------------------------------
% Comentário adicional do PPgSI - Informações sobre ``quadro''
% 
% Caption(Título) de tabelas e ilustração (tais como figura, gráfico, 
% algoritmo, fotografia, quadro etc.) sempre acima da própria.
%
% Para todos os captions/(títulos) (de seções, subseções, tabelas, 
% ilustrações, etc.):
%     - em maiúscula apenas a primeira letra da sentença (do título), 
%       exceto nomes próprios, geográficos, institucionais ou Programas ou
%       Projetos ou siglas, os quais podem ter letras em maiúscula também.
%
% Todas  as tabelas, ilustrações (figuras, quadros, gráficos etc.), 
% anexos, apêndices devem obrigatoriamente ser citados no texto.
%      - a citação deve vir sempre antes da primeira vez em que a tabela, 
%        ilustração etc., aparecer pela primeira vez.
%
% Não confundir ``tabela'' com ``quadro''. Uma tabela deve ter dados 
% numéricos como informação central. Outros tipos de organização de 
% informações devem ser apresentados em quadros, que é um dos tipos de 
% ilustração. A formatação de um quadro é muito parecida a de uma tabela, 
% porém todos os traços horizontais e verticais devem ser apresentados.
%
%-------------------------------------------------------------------------

%\begin{quadro}[H]
%	\centering
%	\caption{Exemplo de título de quadro}
%	\begin{tabular}{|p{1in} | p{1in} | p{1in} | p{1in} |} \hline
%		
%		Cabeçalho 1	& Cabeçalho 2	& Cabeçalho 3	& Cabeçalho 4 \\ \hline
%		Texto	& texto & texto	& texto \\ \hline
%		Texto	& texto & texto	& texto \\ \hline
%		Texto	& texto & texto	& texto \\ \hline
%		Texto	& texto & texto	& texto \\ \hline
%		Texto	& texto & texto	& texto \\ \hline
%		
%	\end{tabular}
%	\label{qua:ExemploDeQuadro1}
%	\source{Marcelo Fantinato, 2015}
%\end{quadro}



\documentclass[
	% -- opções da classe memoir --
	12pt,				% tamanho da fonte
	% openright,			% capítulos começam em pág ímpar (insere página vazia caso preciso)
	oneside,			% para impressão apenas no anverso (apenas frente). Oposto a twoside
	a4paper,			% tamanho do papel. 
	% -- opções da classe abntex2 --
	%chapter=TITLE,		% títulos de capítulos convertidos em letras maiúsculas
	%section=TITLE,		% títulos de seções convertidos em letras maiúsculas
	%subsection=TITLE,	% títulos de subseções convertidos em letras maiúsculas
	%subsubsection=TITLE,% títulos de subsubseções convertidos em letras maiúsculas
	% -- opções do pacote babel --
	brazil,			% idioma adicional para hifenização
	%french,				% idioma adicional para hifenização
	%spanish,			% idioma adicional para hifenização
	english				% o último idioma é o principal do documento
	]{abntex2ppgsi}

% ---
% Pacotes básicos 
% ---
% \usepackage{lmodern}			% Usa a fonte Latin Modern			
% \usepackage[T1]{fontenc}		% Selecao de codigos de fonte.
\usepackage[utf8]{inputenc}		% Codificacao do documento (conversão automática dos acentos)
\usepackage{lastpage}			% Usado pela Ficha catalográfica
\usepackage{indentfirst}		% Indenta o primeiro parágrafo de cada seção.
\usepackage{color}				% Controle das cores
\usepackage{graphicx}			% Inclusão de gráficos
\usepackage{microtype} 			% para melhorias de justificação
\usepackage{pdfpages}    		%para incluir pdf
\usepackage{algorithm}			%para ilustrações do tipo algoritmo
\usepackage{mdwlist}			%para itens com espaço padrão da abnt
\usepackage[noend]{algpseudocode}			%para ilustrações do tipo algoritmo
\usepackage{amsmath,amssymb,amsfonts,amsthm,exscale} %Pacotes matematicos	
		
% ---
% Pacotes adicionais, usados apenas no âmbito do Modelo Canônico do abnteX2
% ---
\usepackage{lipsum}				% para geração de dummy text
% ---

% ---
% Pacotes de citações
% ---
\usepackage[english,hyperpageref]{backref}	 % Paginas com as citações na bibl
\usepackage[alf]{abntex2cite}	% Citações padrão ABNT


% --- 
% CONFIGURAÇÕES DE PACOTES
% --- 

% ---
% Configurações do pacote backref
% Usado sem a opção hyperpageref de backref
\renewcommand{\backrefpagesname}{Citado na(s) página(s):~}
% Texto padrão antes do número das páginas
\renewcommand{\backref}{}
% Define os textos da citação
\renewcommand*{\backrefalt}[4]{
	\ifcase #1 %
		Nenhuma citação no texto.%
	\or
		Citado na página #2.%
	\else
		Citado #1 vezes nas páginas #2.%
	\fi}%
% ---

% ---
% Informações de dados para CAPA e FOLHA DE ROSTO
% ---

%-------------------------------------------------------------------------
% Comentário adicional do PPgSI - Informações sobxre o ``instituicao'':
%
% Não mexer. Deixar exatamente como está.
%
%-------------------------------------------------------------------------
\instituicao{
	Universidade Federal de Minas Gerais
	\par
	Escola de Engenharia
	\par
	Programa de Pós-Graduação em Engenharia Elétrica
	\par
	Laboratório de Modelagem, Análise e Controle de Sistemas Não-Lineares}

%-------------------------------------------------------------------------
% Comentário adicional do PPgSI - Informações sobre o ``título'':
%
% Em maiúscula apenas a primeira letra da sentença (do título), exceto 
% nomes próprios, geográficos, institucionais ou Programas ou Projetos ou 
% siglas, os quais podem ter letras em maiúscula também.
%
% O subtítulo do trabalho é opcional.
% Sem ponto final.
%
% Atenção: o título da Dissertação na versão corrigida não pode mudar. 
% Ele deve ser idêntico ao da versão original.
%
%-------------------------------------------------------------------------
\titulo{Sensor Fusion for Irregular Sampled Systems}

%-------------------------------------------------------------------------
% Comentário adicional do PPgSI - Informações sobre o ``autor'':
%
% Todas as letras em maiúsculas.
% Nome completo.
% Sem ponto final.
%-------------------------------------------------------------------------
\autor{Taiguara Melo Tupinambás}

%-------------------------------------------------------------------------
% Comentário adicional do PPgSI - Informações sobre o ``local'':
%
% Não incluir o ``estado''.
% Sem ponto final.
%-------------------------------------------------------------------------
\local{Belo Horizonte - MG}

%-------------------------------------------------------------------------
% Comentário adicional do PPgSI - Informações sobre a ``data'':
%
% Colocar o ano do depósito (ou seja, o ano da entrega) da respectiva 
% versão, seja ela a versão original (para a defesa) seja ela a versão 
% corrigida (depois da aprovação na defesa). 
%
% Atenção: Se a versão original for depositada no final do ano e a versão 
% corrigida for entregue no ano seguinte, o ano precisa ser atualizado no 
% caso da versão corrigida. 
% Cuidado, pois o ano da ``capa externa'' também precisa ser atualizado 
% nesse caso.
%
% Não incluir o dia, nem o mês.
% Sem ponto final.
%-------------------------------------------------------------------------
\data{2018}

%-------------------------------------------------------------------------
% Comentário adicional do PPgSI - Informações sobre o ``Orientador'':
%
% Se for uma professora, trocar por ``Profa. Dra.''
% Nome completo.
% Sem ponto final.
%-------------------------------------------------------------------------
\orientador{Prof. Dr. Bruno Otávio Soares Teixeira}

%-------------------------------------------------------------------------
% Comentário adicional do PPgSI - Informações sobre o ``Coorientador'':
%
% Opcional. Incluir apenas se houver co-orientador formal, de acordo com o 
% Regulamento do Programa.
%
% Se for uma professora, trocar por ``Profa. Dra.''
% Nome completo.
% Sem ponto final.
%-------------------------------------------------------------------------
\coorientador{Prof. Dr. Leonardo Antônio Borges Tôrres}

\tipotrabalho{Dissertação (Mestrado)}


\preambulo{Dissertação apresentada ao Programa de Pós-Graduação em Engenharia Elétrica da Universidade Federal de Minas Gerais como requisito parcial para a obtenção do grau de Mestre em Engenharia Elétrica}



% ---
% Configurações de aparência do PDF final

% alterando o aspecto da cor azul
\definecolor{blue}{RGB}{41,5,195}

% informações do PDF
\makeatletter
\hypersetup{
     	%pagebackref=true,
		pdftitle={\@title}, 
		pdfauthor={\@author},
    	pdfsubject={\imprimirpreambulo},
	    pdfcreator={LaTeX com abnTeX2 adaptado para o PPgSI-EACH-USP},
		pdfkeywords={abnt}{latex}{abntex}{abntex2}{qualificação de mestrado}{dissertação de mestrado}{ppgsi}, 
		colorlinks=true,       		% false: boxed links; true: colored links
    	linkcolor=black,          	% color of internal links
    	citecolor=black,        		% color of links to bibliography
    	filecolor=black,      		% color of file links
		urlcolor=black,
		bookmarksdepth=4
}


\makeatother
% --- 

% --- 
% Espaçamentos entre linhas e parágrafos 
% --- 

% O tamanho do parágrafo é dado por:
\setlength{\parindent}{1.25cm}

% Controle do espaçamento entre um parágrafo e outro:
\setlength{\parskip}{0cm}  % tente também \onelineskip
\renewcommand{\baselinestretch}{1.5}

% ---
% compila o indice
% ---
\makeindex
% ---

	% Controlar linhas orfas e viuvas
  \clubpenalty10000
  \widowpenalty10000
  \displaywidowpenalty10000

% ----
% Início do documento
% ----
\begin{document}

% Retira espaço extra obsoleto entre as frases.
\frenchspacing 

% ----------------------------------------------------------
% ELEMENTOS PRÉ-TEXTUAIS
% ----------------------------------------------------------
% \pretextual

% ---
% Capa
% ---
%-------------------------------------------------------------------------
% Comentário adicional do PPgSI - Informações sobre a ``capa'':
%
% Esta é a ``capa'' principal/oficial do trabalho, a ser impressa apenas 
% para os casos de encadernação simples (ou seja, em ``espiral'' com 
% plástico na frente).CCC
% 
% Não imprimir esta ``capa'' quando houver ``capa dura'' ou ``capa brochura'' 
% em que estas mesmas informações já estão presentes nela.
%
%-------------------------------------------------------------------------
\imprimircapa
% ---	



% ---
% ---
%% inserir lista de figuras
%% ---Cgurename}{lof}
%\listoffigures*
%\cleardoublepage
% ---

% ---
%% inserir lista de tabelas
%% ---
%\pdfbookmark[0]{\listtablename}{lot}
%\listoftables*
%\cleardoublepage
% ---


% ---
% inserir o sumario
% ---
\pdfbookmark[0]{\contentsname}{toc}
\tableofcontents*
\cleardoublepage
% ---



% ----------------------------------------------------------
% ELEMENTOS TEXTUAIS
% ----------------------------------------------------------
\textual

%-------------------------------------------------------------------------
% Comentário adicional do PPgSI - Informações sobre ``títulos de seções''
% 
% Para todos os títulos (seções, subseções, tabelas, ilustrações, etc.):
%
% Em maiúscula apenas a primeira letra da sentença (do tít		ulo), exceto 
% nomes próprios, geográficos, institucionais ou Programas ou Projetos ou
% siglas, os quais podem ter letras em maiúscula também.
%
%-------------------------------------------------------------------------
\chapter{Introduction}

%-------------------------------------------------------------------------

\section{Motivations}

Nature has learned to merge multiple information from several sources a long time ago, in order to have a better perception of the environment. Animals combine signals received by different senses, such as sight, hearing, smell, taste and touch to recognize the surroundings. Plants have an analogous sensory system, used to water consumption modulation, leaf-colour changes or structure bending towards the light, for instance. Throughout history evolution has led to highly complex and efficient muli-sensor systems in living beings. 

Nowadays information fusion is studied in many fields of science, as a way of exploiting information from multiple sources to achieve better outcomes in comparison to those obtained if any of theses sources were used separately \cite{Dasarathy2001}. Other terms have been used to denote the merge of information in technical literature, e.g. "data fusion", "sensor fusion", "multi-sensor fusion" or "multi-sensor integration". To avoid confusion, the terminology used by \cite{Elmenreich2002} will be adopted, whereby information fusion is understood as the overall term and sensor fusion is used in case the sources of information are sensors. 

Some research fields have been taken increasingly advantage of sensor fusion techniques, such as robotics, military, biometrics and image processing. The main benefits expected are related to accuracy, due to the use of redundant or complimentary data, dimensionality, i.e. additional information being created by a group of data and robustness against failure and interference.

The most common application of sensor fusion is to estimate states of a process, based on the system dynamics, the measurement model, the disturbances and the prior information \cite{Bar-Shalom2001}. If the system is linear and Gaussian, the Kalman filter (KF) guarantees optimal estimation. For nonlinear processes, KF generalizations were proposed, such as the extended Kalman filter (EKF) or the unscented Kalman filter (UKF). Particle filters (PF) on the other hand can be used to deal with both nonlinearities in the dynamics and non-Gaussian distributions.



\section{Problem Formulation}

Apresentação matemática do problema, de forma ampla. Descrever as premissas adotadas.

\section{Objectives}

1 frase para o objetivo geral
Objetivos específicos

\section{Thesis Structure}

\chapter{Literature Review}

\section{Multi-Sensor Systems}

\section{Irregularly Sampled Systems}

\section{State Estimation with Irregular Sampling}

\subsection{Known Measurement Instance}

\subsection{Unknown Measurement Instance}

\chapter{Methods}

\section{Unscented Kalman Filter}

\section{Estimation With Timestamp}

\section{Estimation Without Timestamp}

\chapter{Results}

\section{Unicycle Position Estimation}

\subsection{Measurement Signal-to-Noise Ratio Variation}

\subsection{Average Sampling Rate Variation}

\subsection{Regular and Average Irregular Time Interval Relation Variation}

\chapter{Conclusion}

% ----------------------------------------------------------
% ELEMENTOS PÓS-TEXTUAIS
% ----------------------------------------------------------
\postextual
% ----------------------------------------------------------

% ----------------------------------------------------------
% Referências bibliográficas
% ----------------------------------------------------------
\bibliography{referencias}

% ----------------------------------------------------------
% Glossário
% ----------------------------------------------------------
%
% Consulte o manual da classe abntex2 para orientações sobre o glossário.
%
%\glossary

% ----------------------------------------------------------
% Apêndices
% ----------------------------------------------------------

% ---
% Inicia os apêndices
% ---
%\begin{apendicesenv}

% Imprime uma página indicando o início dos apêndices
%\partapendices

%-------------------------------------------------------------------------
% Comentário adicional do PPgSI - Informações sobre ``apêndice''
%
% Para todos os captions/(títulos) (de seções, subseções, tabelas, 
% ilustrações, etc.):
%     - em maiúscula apenas a primeira letra da sentença (do título), 
%       exceto nomes próprios, geográficos, institucionais ou Programas ou
%       Projetos ou siglas, os quais podem ter letras em maiúscula também.
%
% Todas  as tabelas, ilustrações (figuras, quadros, gráficos etc. ), 
% anexos, apêndices devem obrigatoriamente ser citados no texto.
%      - a citação deve vir sempre antes da primeira vez em que a tabela, 
%        ilustração etc., aparecer pela primeira vez.
%
%%-------------------------------------------------------------------------
%\chapter{Exemplo de apêndice}
%
%Texto de exemplo, texto de exemplo, texto de exemplo, texto de exemplo, texto de exemplo, texto de exemplo, texto de exemplo, texto de exemplo, texto de exemplo, texto de exemplo, texto de exemplo, texto de exemplo, texto de exemplo, texto de exemplo, texto de exemplo, texto de exemplo, texto de exemplo, texto de exemplo, texto de exemplo.
%
%\section*{1 Exemplo de seção de apêndice não apresentada no sumário}
%
%Texto de exemplo, texto de exemplo, texto de exemplo, texto de exemplo, texto de exemplo, texto de exemplo, texto de exemplo, texto de exemplo, texto de exemplo, texto de exemplo, texto de exemplo, texto de exemplo, texto de exemplo, texto de exemplo, texto de exemplo, texto de exemplo, texto de exemplo, texto de exemplo, texto de exemplo.
%
%\subsection*{1.1 Exemplo de subseção de apêndice não apresentada no sumário}
%
%Texto de exemplo, texto de exemplo, texto de exemplo, texto de exemplo, texto de exemplo, texto de exemplo, texto de exemplo, texto de exemplo, texto de exemplo, texto de exemplo, texto de exemplo, texto de exemplo, texto de exemplo, texto de exemplo, texto de exemplo, texto de exemplo, texto de exemplo, texto de exemplo, texto de exemplo.
%
%\subsection*{1.2 Exemplo de subseção de apêndice não apresentada no sumário}
%
%Texto de exemplo, texto de exemplo, texto de exemplo, texto de exemplo, texto de exemplo, texto de exemplo, texto de exemplo, texto de exemplo, texto de exemplo, texto de exemplo, texto de exemplo, texto de exemplo, texto de exemplo, texto de exemplo, texto de exemplo, texto de exemplo, texto de exemplo, texto de exemplo, texto de exemplo.
%
%\subsection*{1.3 Exemplo de subseção de apêndice não apresentada no sumário}
%
%Texto de exemplo, texto de exemplo, texto de exemplo, texto de exemplo, texto de exemplo, texto de exemplo, texto de exemplo, texto de exemplo, texto de exemplo, texto de exemplo, texto de exemplo, texto de exemplo, texto de exemplo, texto de exemplo, texto de exemplo, texto de exemplo, texto de exemplo, texto de exemplo, texto de exemplo.
%
%\section*{2 Exemplo de seção de apêndice não apresentada no sumário}
%
%Texto de exemplo, texto de exemplo, texto de exemplo, texto de exemplo, texto de exemplo, texto de exemplo, texto de exemplo, texto de exemplo, texto de exemplo, texto de exemplo, texto de exemplo, texto de exemplo, texto de exemplo, texto de exemplo, texto de exemplo, texto de exemplo, texto de exemplo, texto de exemplo, texto de exemplo.
%
%\section*{3 Exemplo de seção de apêndice não apresentada no sumário}
%
%Texto de exemplo, texto de exemplo, texto de exemplo, texto de exemplo, texto de exemplo, texto de exemplo, texto de exemplo, texto de exemplo, texto de exemplo, texto de exemplo, texto de exemplo, texto de exemplo, texto de exemplo, texto de exemplo, texto de exemplo, texto de exemplo, texto de exemplo, texto de exemplo, texto de exemplo.
%
%
%\chapter{Exemplo de apêndice}
%
%Texto de exemplo, texto de exemplo, texto de exemplo, texto de exemplo, texto de exemplo, texto de exemplo, texto de exemplo, texto de exemplo, texto de exemplo, texto de exemplo, texto de exemplo, texto de exemplo, texto de exemplo, texto de exemplo, texto de exemplo, texto de exemplo, texto de exemplo, texto de exemplo, texto de exemplo.
%
%\section*{1 Exemplo de seção de apêndice não apresentada no sumário}
%
%Texto de exemplo, texto de exemplo, texto de exemplo, texto de exemplo, texto de exemplo, texto de exemplo, texto de exemplo, texto de exemplo, texto de exemplo, texto de exemplo, texto de exemplo, texto de exemplo, texto de exemplo, texto de exemplo, texto de exemplo, texto de exemplo, texto de exemplo, texto de exemplo, texto de exemplo.
%
%\subsection*{1.1 Exemplo de subseção de apêndice não apresentada no sumário}
%
%Texto de exemplo, texto de exemplo, texto de exemplo, texto de exemplo, texto de exemplo, texto de exemplo, texto de exemplo, texto de exemplo, texto de exemplo, texto de exemplo, texto de exemplo, texto de exemplo, texto de exemplo, texto de exemplo, texto de exemplo, texto de exemplo, texto de exemplo, texto de exemplo, texto de exemplo.
%
%\subsection*{1.2 Exemplo de subseção de apêndice não apresentada no sumário}
%
%Texto de exemplo, texto de exemplo, texto de exemplo, texto de exemplo, texto de exemplo, texto de exemplo, texto de exemplo, texto de exemplo, texto de exemplo, texto de exemplo, texto de exemplo, texto de exemplo, texto de exemplo, texto de exemplo, texto de exemplo, texto de exemplo, texto de exemplo, texto de exemplo, texto de exemplo.
%
%\subsection*{1.3 Exemplo de subseção de apêndice não apresentada no sumário}
%
%Texto de exemplo, texto de exemplo, texto de exemplo, texto de exemplo, texto de exemplo, texto de exemplo, texto de exemplo, texto de exemplo, texto de exemplo, texto de exemplo, texto de exemplo, texto de exemplo, texto de exemplo, texto de exemplo, texto de exemplo, texto de exemplo, texto de exemplo, texto de exemplo, texto de exemplo.
%
%\section*{2 Exemplo de seção de apêndice não apresentada no sumário}
%
%Texto de exemplo, texto de exemplo, texto de exemplo, texto de exemplo, texto de exemplo, texto de exemplo, texto de exemplo, texto de exemplo, texto de exemplo, texto de exemplo, texto de exemplo, texto de exemplo, texto de exemplo, texto de exemplo, texto de exemplo, texto de exemplo, texto de exemplo, texto de exemplo, texto de exemplo.
%
%\section*{3 Exemplo de seção de apêndice não apresentada no sumário}
%
%Texto de exemplo, texto de exemplo, texto de exemplo, texto de exemplo, texto de exemplo, texto de exemplo, texto de exemplo, texto de exemplo, texto de exemplo, texto de exemplo, texto de exemplo, texto de exemplo, texto de exemplo, texto de exemplo, texto de exemplo, texto de exemplo, texto de exemplo, texto de exemplo, texto de exemplo.
%
%
%\chapter{Exemplo de apêndice}
%
%Texto de exemplo, texto de exemplo, texto de exemplo, texto de exemplo, texto de exemplo, texto de exemplo, texto de exemplo, texto de exemplo, texto de exemplo, texto de exemplo, texto de exemplo, texto de exemplo, texto de exemplo, texto de exemplo, texto de exemplo, texto de exemplo, texto de exemplo, texto de exemplo, texto de exemplo.
%
%\section*{1 Exemplo de seção de apêndice não apresentada no sumário}
%
%Texto de exemplo, texto de exemplo, texto de exemplo, texto de exemplo, texto de exemplo, texto de exemplo, texto de exemplo, texto de exemplo, texto de exemplo, texto de exemplo, texto de exemplo, texto de exemplo, texto de exemplo, texto de exemplo, texto de exemplo, texto de exemplo, texto de exemplo, texto de exemplo, texto de exemplo.
%
%\subsection*{1.1 Exemplo de subseção de apêndice não apresentada no sumário}
%
%Texto de exemplo, texto de exemplo, texto de exemplo, texto de exemplo, texto de exemplo, texto de exemplo, texto de exemplo, texto de exemplo, texto de exemplo, texto de exemplo, texto de exemplo, texto de exemplo, texto de exemplo, texto de exemplo, texto de exemplo, texto de exemplo, texto de exemplo, texto de exemplo, texto de exemplo.
%
%\subsection*{1.2 Exemplo de subseção de apêndice não apresentada no sumário}
%
%Texto de exemplo, texto de exemplo, texto de exemplo, texto de exemplo, texto de exemplo, texto de exemplo, texto de exemplo, texto de exemplo, texto de exemplo, texto de exemplo, texto de exemplo, texto de exemplo, texto de exemplo, texto de exemplo, texto de exemplo, texto de exemplo, texto de exemplo, texto de exemplo, texto de exemplo.
%
%\subsection*{1.3 Exemplo de subseção de apêndice não apresentada no sumário}
%
%Texto de exemplo, texto de exemplo, texto de exemplo, texto de exemplo, texto de exemplo, texto de exemplo, texto de exemplo, texto de exemplo, texto de exemplo, texto de exemplo, texto de exemplo, texto de exemplo, texto de exemplo, texto de exemplo, texto de exemplo, texto de exemplo, texto de exemplo, texto de exemplo, texto de exemplo.
%
%\section*{2 Exemplo de seção de apêndice não apresentada no sumário}
%
%Texto de exemplo, texto de exemplo, texto de exemplo, texto de exemplo, texto de exemplo, texto de exemplo, texto de exemplo, texto de exemplo, texto de exemplo, texto de exemplo, texto de exemplo, texto de exemplo, texto de exemplo, texto de exemplo, texto de exemplo, texto de exemplo, texto de exemplo, texto de exemplo, texto de exemplo.
%
%\section*{3 Exemplo de seção de apêndice não apresentada no sumário}
%
%Texto de exemplo, texto de exemplo, texto de exemplo, texto de exemplo, texto de exemplo, texto de exemplo, texto de exemplo, texto de exemplo, texto de exemplo, texto de exemplo, texto de exemplo, texto de exemplo, texto de exemplo, texto de exemplo, texto de exemplo, texto de exemplo, texto de exemplo, texto de exemplo, texto de exemplo.
%
%
%
%
%\end{apendicesenv}
%% ---
%
%
%% ----------------------------------------------------------
%% Anexos
%% ----------------------------------------------------------
%
%% ---
%% Inicia os anexos
%% ---
%\begin{anexosenv}
%
%% Imprime uma página indicando o início dos anexos
%%\partanexos
%
%
%%-------------------------------------------------------------------------
%% Comentário adicional do PPgSI - Informações sobre ``anexo''
%%
%% Para todos os captions/(títulos) (de seções, subseções, tabelas, 
%% ilustrações, etc.):
%%     - em maiúscula apenas a primeira letra da sentença (do título), 
%%       exceto nomes próprios, geográficos, institucionais ou Programas ou
%%       Projetos ou siglas, os quais podem ter letras em maiúscula também.
%%
%% Todas  as tabelas, ilustrações (figuras, quadros, gráficos etc. ), 
%% anexos, apêndices devem obrigatoriamente ser citados no texto.
%%      - a citação deve vir sempre antes da primeira vez em que a tabela, 
%%        ilustração etc., aparecer pela primeira vez.
%%
%%-------------------------------------------------------------------------
%\chapter{Resumo das normas}
%\label{anexoA}
%
%Considerando a dificuldade para formatar um texto acadêmico sem conhecimento básico do conteúdo da norma NBR 14724 ``Informação e documentação – Trabalhos acadêmicos – Apresentação'', este anexo apresenta um resumo de alguns conceitos dessa norma, conforme publicada em julho de 2011. Sugere-se a leitura completa da norma para garantir que seu documento seja completamente aderente à mesma. Em alguns casos específicos, este anexo apresenta alguns ajustes da norma especificamente para o PPgSI.
%
%\section*{1 NBR 14724: estrutura e algumas descrições}
%
%A estrutura de uma tese, dissertação ou qualquer outro trabalho acadêmico, deve compreender elementos pré-textuais, elementos textuais e elementos pós-textuais, que aparecem no texto na seguinte ordem:
%
%\subsection*{1.1 Elementos pré-textuais}
%
%\begin{itemize}
%	\item Capa (obrigatório)
%	\item	Folha de rosto (obrigatório)
%	\item	Errata (opcional)
%	\item	Folha de aprovação (obrigatório)
%	\item	Dedicatória (opcional)
%	\item	Agradecimentos (opcional)
%	\item	Epígrafe (opcional)
%	\item	Resumo em língua vernácula (obrigatório)
%	\item	Resumo em língua estrangeira (obrigatório)
%	\item	Listas de ilustrações: lista de figuras, lista de algoritmos, lista de quadros etc. (opcional)
%	\item	Lista de tabelas (opcional)
%	\item	Lista de abreviaturas e siglas (opcional)
%	\item	Lista de símbolos (opcional)
%	\item	Sumário (obrigatório)
%\end{itemize}
%
%\subsection*{1.2 Elementos textuais}
%
%\begin{itemize}
%	\item	Introdução
%	\item	Desenvolvimento
%	\item	Conclusão
%\end{itemize}
%
%\subsection*{1.3 Elementos pós-textuais}
%
%\begin{itemize}
%	\item	Referências (obrigatório)
%	\item	Apêndice (opcional)
%	\item	Anexo (opcional)
%	\item	Glossário (opcional)
%\end{itemize}
%
%\section*{2 Definições relacionadas a elementos pré-textuais}
%
%A seguir, são apresentadas algumas definições contidas na norma relacionadas a elementos pré-textuais.
%
%\subsection*{2.1 Capa}
%
%Elemento obrigatório, para proteção externa e sobre o qual se imprimem informações que ajudam na identificação e uso do trabalho, na seguinte ordem:
%\begin{enumerate}
%	\item Nome completo do autor: responsável intelectual do trabalho.
%	\item	Título principal do trabalho: deve ser claro e preciso, identificando o seu conteúdo e possibilitando a indexação e recuperação da informação.
%	\item	Subtítulo (se houver): deve ser evidenciada sua subordinação ao título principal, precedido de dois pontos (:).
%	\item	Número do volume (obrigatório apenas se houver mais de um volume, de forma que deve constar em cada capa a especificação do respectivo volume).
%	\item	Local (cidade) da instituição de apresentação.
%	\item	Ano do depósito (entrega).
%\end{enumerate}
%
%\subsection*{2.2 Folha de rosto (anverso)}
%
%Os elementos do anverso da folha de rosto devem figurar na seguinte ordem:
%\begin{enumerate}
%	\item	Nome completo do autor: responsável intelectual do trabalho.
%	\item	Título principal do trabalho: deve ser claro e preciso, identificando o seu conteúdo e possibilitando a indexação e recuperação da informação.
%	\item	Subtítulo (se houver): deve ser evidenciada sua subordinação ao título principal, precedido de dois pontos (:).
%	\item	Número do volume (obrigatório apenas se houver mais de um volume, de forma que deve constar em cada capa a especificação do respectivo volume).
%	\item	Natureza (tese, dissertação e outros) e objetivo (aprovação em disciplina, grau pretendido e outros); nome da instituição a que é submetido; área de concentração.
%	\item	Nome do orientador e, se houver, do co-orientador.
%	\item	Local (cidade) da instituição de apresentação.
%	\item	Ano de depósito (entrega).
%\end{enumerate}
%
%\subsection*{2.3 Folha de rosto (verso)}
%
%No verso da folha de rosto deve constar a ficha catalográfica, conforme o Código de Catalogação Anglo-Americano – CCAA2.
%
%\subsection*{2.4 Folha de aprovação}
%
%Elemento obrigatório, que contém autor, título por extenso e subtítulo, se houver, local e data de aprovação, nome e instituição dos membros componentes da banca examinadora.
%
%\subsection*{2.5 Dedicatória e agradecimentos}
%
%Elementos opcionais. Os agradecimentos devem ser dirigidos apenas àqueles que contribuíram de maneira relevante à elaboração do trabalho.
%
%\subsection*{2.6 Resumo na língua vernácula}
%
%Elemento obrigatório, que consiste na apresentação concisa dos pontos relevantes de um texto; constitui-se em uma sequência de frases concisas e objetivas, e não de uma simples enumeração de tópicos, não ultrapassando 500 palavras, seguido, logo abaixo, das palavras representativas do conteúdo do trabalho, isto é, palavras-chave e/ou descritores.
%
%\subsection*{2.7 Resumo em língua estrangeira}
%
%Elemento obrigatório, que consiste em uma versão do resumo em idioma de divulgação internacional (em inglês Abstract, em castelhano Resumen, em francês Résumé, por exemplo). Deve ser seguido das palavras representativas do conteúdo do trabalho, isto é, palavras-chave e/ou descritores, na respectiva língua estrangeira.
%
%\subsection*{2.8 Lista de figuras e lista de tabelas}
%
%Elementos opcionais, elaborados de acordo com a ordem apresentada no texto, com cada item acompanhado do respectivo número da página.
%
%\subsection*{2.9 Lista de abreviaturas e siglas}
%
%Elemento opcional. Consiste na relação alfabética das abreviaturas e siglas usadas no texto, seguidas das palavras ou expressões correspondentes grafadas por extenso.
%
%\subsection*{2.10 Lista de símbolos}
%
%Elemento opcional, elaborado de acordo com a ordem apresentada no texto, com o devido significado.
%
%\subsection*{2.11 Sumário}
%
%Elemento obrigatório, que consiste na enumeração das principais divisões (seções e outras partes do trabalho) dos elementos textuais e pós-textuais, na mesma ordem e grafia em que a matéria nele sucede, acompanhado do respectivo número da página.
%
%\section*{3 Definições relacionadas a elementos textuais}
%
%O autor deve criar quantas seções primárias (também chamadas informalmente de capítulos) desejar para tratar dos seguintes elementos textuais que são obrigatórios: introdução, desenvolvimento e conclusão. Normalmente, existe apenas uma seção primária para a introdução, uma ou mais seções primárias para o desenvolvimento, e apenas uma seção primária para a conclusão.
%
%\section*{4 Definições relacionadas a elementos pós-textuais}
%
%A seguir, são apresentadas algumas definições contidas na norma relacionadas a elementos pós-textuais.
%
%\subsection*{4.1 Apêndice}
%
%Elemento opcional, que consiste em um texto ou documento elaborado pelo próprio autor, a fim de complementar sua argumentação, sem prejuízo da unidade nuclear do trabalho. Um apêndice deve ser identificado por uma letra maiúscula, seguida por um hífen (entre caracteres de espaço), seguido pelo respectivo título. Os apêndices devem ser identificados por letras consecutivas, a partir da letra ``A'' (independentemente dos anexos).
%
%\subsection*{4.2 Anexo}
%
%Elemento opcional, que consiste em um texto ou documento não elaborado pelo autor, a fim de fundamentar, comprovar ou ilustrar a argumentação do autor. Um anexo deve ser identificado por uma letra maiúscula, seguida por um hífen (entre caracteres de espaço), seguido pelo respectivo título. Os anexos devem ser identificados por letras consecutivas, a partir da letra ``A'' (independentemente dos apêndices).
%
%\subsection*{4.3 Glossário}
%
%Elemento opcional, que consiste em uma lista em ordem alfabética de palavras ou de expressões técnicas de uso restrito ou de sentido obscuro, usadas no texto, acompanhadas das respectivas definições.
%
%\section*{5 Formas de apresentação}
%
%A seguir, são apresentadas algumas definições contidas na norma relacionadas a formas de apresentação em geral.
%
%\subsection*{5.1 Formato}
%
%O texto deve estar impresso em papel branco, formato A4 (21,0 cm 29,7 cm), apenas no anverso da folha (ou seja, na ``frente'' da folha), excetuando-se a folha de rosto que deve estar impressa tanto no anverso quanto no verso (com a ficha catalográfica).
%
%\subsection*{5.2 Projeto gráfico}
%
%O projeto gráfico é de responsabilidade do autor.
%
%\subsection*{5.3 Fonte}
%
%
%Usar sempre cor preta.
%
%Usar sempre tamanho de fonte 12, com as seguintes exceções: tamanho de fonte 10 para citações longas (com mais de três linhas), notas de rodapé, legendas de ilustração e de tabela, fontes de ilustração e de tabela, números de página; e tamanho de fonte maiores para títulos de seção (conforme apresentado na seção 6.1 a seguir).
%
%\subsection*{5.4 Margens}
%
%Todas as folhas devem apresentar margens esquerda e superior de 3 cm; e margens direita e inferior de 2 cm, considerando impressão apenas no anverso (ou seja, apenas na ``frente''). 
%
%Se a impressão precisar, por algumo motivo especial, ser realizada em anverso e verso (ou seja, em frente e verso), neste caso, há que se configurar as margens de forma diferente, conforme detalhes da norma ABNT, além de outros detalhes de configuração; por isso solicita-se não realizar impressão em frente e verso.
%
%\subsection*{5.5 Espaçamento entre linhas}
%
%Usar sempre espaçamento entre linhas de 1,5 linhas, com as seguintes exceções: espaçamento entre linhas ``simples'' para citações longas (com mais de três linhas), notas de rodapé, referências, resumos (em vernáculo e em língua estrangeira), legendas de ilustração e de tabela, fontes de ilustração e de tabela, ficha catalográfica, natureza do trabalho, grau pretendido, nome da instituição a que é submetido, e área de concentração; e espaçamento entre linhas ``duplo'' para equações e fórmulas e para separação das referências entre si.
%
%Os títulos das seções devem começar na margem superior da folha separados do texto que os sucede por um espaço em branco de 1,5 e, da mesma forma, os títulos das subseções devem ser separados do texto que os precede, ou que os sucede, por um espaço em branco de 1,5.
%
%\subsection*{5.6 Numeração das seções}
%
%O indicativo numérico de uma seção precede seu título, alinhado à esquerda, separado por um espaço de caractere. Nos títulos sem indicativo numérico, como lista de ilustrações, sumário, resumo, referências e outros, devem ser centralizados.
%
%Para evidenciar a sistematização do conteúdo do trabalho, deve-se adotar a numeração progressiva para as seções do texto. Os títulos das seções primárias (chamadas informalmente de capítulos), por serem as principais divisões do texto, devem iniciar em folha distinta. Títulos das seções e subseções devem ser destacados gradativamente, usando-se os recursos de negrito, itálico ou grifo e redondo, caixa alta ou versal.
%
%\subsection*{5.7 Paginação}
%
%Todas as folhas do trabalho, a partir da folha de rosto (desconsiderando a capa, mas considerando a ficha catalográfica), devem ser contadas sequencialmente, mas não numeradas. A numeração é colocada, a partir da primeira folha da dos elementos textuais (ou seja, a partir da ``Introdução''), em algarismos arábicos, no canto superior direito da folha, a 2 cm da borda superior, ficando o último algarismo a 2 cm da borda direita da folha.
%
%Havendo apêndices e/ou anexos, suas folhas devem ser numeradas de maneira contínua e sua paginação deve dar seguimento à do texto principal, em algarismos arábicos.
%
%No caso de o trabalho ser constituído de mais de um volume, deve-se manter uma única sequência de numeração das folhas, do primeiro ao último volume.
%
%\subsection*{5.8 Equações e fórmulas}
%
%Equações e fórmulas devem aparecer destacadas no texto, para facilitar sua leitura. 
%
%Se as equações e fórmulas forem apresentadas na sequência normal do texto (ou seja, dentro do próprio parágrafo normal de texto), é permitido usar um espaçamento entre linhas duplo para comportar seus elementos (ou seja, expoentes, índices e outros). 
%
%Se as equações e fórmulas forem apresentadas fora do parágrafo, então elas devem ser centralizadas e, se necessário, devem ser numeradas. Quando fragmentadas em mais de uma linha, por falta de espaço, devem ser interrompidas antes do sinal de igualdade ou depois dos sinais de adição, subtração, multiplicação e divisão.
%
%\subsection*{5.9 Ilustrações}
%
%Cada tipo de ilustração (tais como figura, gráfico, algoritmo, fotografia, quadro, esquema, desenhos, esquemas, fluxogramas, mapa, organograma, planta, retrato, entre outros) tem numeração independente e consecutiva. 
%
%Inserir a ilustração o mais próximo possível do parágrafo em que ela é citada pela primeira vez no texto; nunca inserir uma ilustração antes de ela ser citada pela primeira vez no texto. Toda ilustração inserida no trabalho deve ser citada pelo menos uma vez no texto.
%
%Qualquer que seja o tipo da ilustração, ela deve obrigatoriamente ter uma identificação (ou seja, um título), que deve aparecer sempre na parte superior da ilustração, precedida pela palavra que identifica seu tipo, por exemplo ``Figura'', seguida de seu número de ordem de ocorrência no texto em algarismo arábico, e de um hífen entre caracteres de espaço (`` – ''), em fonte com tamanho 12, sem negrito, sem itálico, com apenas a primeira letra da sentença maiúscula, sem ponto final, e em espaçamento simples. Exemplo: ``Figura 1 – Título da ilustração''.
%
%Para toda ilustração, deve ser apresentada também obrigatoriamente sua fonte (mesmo quando a fonte é o próprio autor do trabalho). A fonte deve apresentada na parte inferior da ilustração e ser informada no seguinte formato: palavra ``Fonte'', seguida pelo caractere dois pontos ``:'', seguido por um caractere de espaço, seguido pela citação de onde a ilustração foi obtida (conforme regras de citação da norma ABNT) ou seguido pelo nome completo do autor do trabalho, por uma vírgula e pelo ano de elaboração do trabalho (caso a ilustração seja de elaboração do próprio autor), em fonte com tamanho 10, sem negrito, sem itálico, sem ponto final, e em espaçamento simples. Exemplo 1 (quando se trata de fonte externa): ``Fonte: citação conforme norma ABNT''; Exemplo 2 (quando se trata do próprio autor do trabalho): ``Fonte: Nome Completo, Ano''.
%
%Para referenciar uma ilustração (por exemplo, do tipo ``figura'') no texto, há duas formas: \textit{(i)} se a referência à figura fizer parte do texto, mesmo que dentro de parênteses, use a palavra ``figura'' com todas as letras em minúsculo, por exemplo – ``A figura 5 apresenta um exemplo de (...)'' ou ``(...) esses dados já foram apresentados na seção anterior (ver figura 5)''; \textit{(ii)} se a referência à figura estiver completamente isolada do texto, dentro de parênteses, use a palavra ``Figura'' com a inicial em maiúsculo, por exemplo ``(...) para um entendimento mais claro, essas informações estão apresentadas graficamente (Figura 5)''.
%
%\subsection*{5.10 Tabelas}
%
%As tabelas têm numeração independente e consecutiva das ilustrações.
%
%Inserir a tabela o mais próximo possível do parágrafo em que ela é citada pela primeira vez no texto; nunca inserir uma tabela antes de ela ser citada pela primeira vez no texto. Toda tabela inserida no trabalho deve ser citada pelo menos uma vez no texto.
%
%Toda tabela deve obrigatoriamente ter uma identificação (ou seja, um título), que deve aparece na parte superior, precedida pela palavra ``Tabela'', seguida de seu número de ordem de ocorrência no texto em algarismo arábico, e de um hífen entre caracteres de espaço (`` – ''), em fonte com tamanho 12, sem negrito, sem itálico, com apenas a primeira letra da sentença maiúscula, sem ponto final, e em espaçamento simples. Exemplo: ``Tabela 1 – Título da tabela''.
%
%Para toda tabela, deve ser apresentada também obrigatoriamente sua fonte (mesmo quando a fonte é o próprio autor do trabalho). A fonte deve apresentada na parte inferior da tabela e ser informada no seguinte formato: palavra ``Fonte:'', seguida pelo caractere dois pontos ``:'', seguido por um caractere de espaço, seguido pela citação de onde a fonte foi obtida (conforme regras de citação da norma ABNT) ou seguido pelo nome completo do autor do trabalho, por uma vírgula e pelo ano de elaboração do trabalho (caso a fonte seja de elaboração do próprio autor), em fonte com tamanho 10, sem negrito, sem itálico, sem ponto final, e em espaçamento simples. Exemplo 1 (quando se trata de fonte externa): ``Fonte: citação conforme norma ABNT''; Exemplo 2 (quando se trata do próprio autor do trabalho): ``Fonte: Nome Completo, Ano''.
%
%Usar traços horizontais apenas para delimitar o cabeçalho da tabela e o início e o fim da tabela. Não usar traços horizontais para separar cada linha de conteúdo da tabela e também não usar traços verticais para separar cada coluna de conteúdo da tabela. 
%
%Se a tabela não couber em uma folha, ela deve ser continuada nas folhas seguintes. Nesse caso, a tabela não deve ser delimitada por traço horizontal na parte inferior nas primeiras folhas (mas sim apenas na última folha em que ela realmente é finalizada), e a legenda e o cabeçalho da tabela devem ser repetidos nas folhas seguintes. Além disso, as folhas devem ter as seguintes indicações: ``continua'' (no fim das primeiras folhas); ``continuação'' (no início das folhas intermediárias, se houver) e ``conclusão'' (no início da última folha).
%
%Para referenciar uma tabela no texto, há duas formas: \textit{(i)} se a referência à tabela fizer parte do texto, mesmo que dentro de parênteses, use a palavra ``tabela'' com todas as letras em minúsculo, por exemplo – ``A tabela 5 apresenta um exemplo de (...)'' ou ``(...) esses dados já foram apresentados na seção anterior (ver tabela 5)''; \textit{(ii)} se a referência à tabela estiver completamente isolada do texto, dentro de parênteses, use a palavra ``Tabela'' com a inicial em maiúsculo, por exemplo ``(...) para um entendimento mais claro, essas informações estão apresentadas graficamente (Tabela 5)''.
%
%Não confundir ``tabela'' com ``quadro''. Uma tabela deve ter dados numéricos como informação central. Outros tipos de organização de informações devem ser apresentados em quadros, que é um dos tipos de ilustração. A formatação de um quadro é muito parecida a de uma tabela, porém todos os traços horizontais e verticais devem ser apresentados.
%
%\section*{6 Outras normas}
%
%\subsection*{6.1 Seções}
%
%As seções primárias são as principais divisões do texto, denominadas informalmente de ``capítulos''. As seções primárias podem ser divididas em seções secundárias; e as secundárias em terciárias, em formatação distinta. Não divida o texto mais do que a terceira ordem; ou seja, evite criar seções de profundidade quatro ou cinco.
%
%Todos títulos, de todas as seções, de todos os níveis, devem ter sempre tamanho 12. O que muda é a formatação, conforme segue abaixo:
%
%A formatação adotada para este \textit{template} em particular é a seguinte:
%
%\begin{itemize}
%	\item Seções primárias: \textbf{negrito}.
%	\item Seções secundárias: \textit{itálico}.
%	\item Seções terciárias: regular.
%	\item Seções quartenárias: [não usar].
%	\item Seções quinárias: [não usar].
%\end{itemize}
%
%São empregados algarismos arábicos na numeração. O ``indicativo'' de uma seção precede o título ou a primeira palavra do texto, se não houver título, separado por um espaço. O indicativo da seção secundária é constituído pelo indicativo da seção primária que a precede seguido do número que lhe foi atribuído na sequência do assunto e separado por ponto. Repete-se o mesmo processo em relação às demais seções. Na leitura, não se lê os pontos (por exemplo: ``2.1.1'' lê-se ``dois um um'').
%
%Os indicativos devem ser citados no texto de acordo com os seguintes exemplos: (...) na seção 4 (...); (...) no capítulo 2 (...); (...) ver 9.2 (...); (...) em 1.1.2.2 parág. 3º [ou] (...) no 3º parágrafo de 1.1.2.2; (...) (Seção 2.1) (...).
%
%\subsection*{6.2 Referências bibliográficas e citações às referências bibliográficas}
%
%A norma é bastante complexa e extensa em relação às regras de referências bibliográficas (cerca de 19 páginas) e citações às referências bibliográficas, não sendo possível fazer um resumo aqui. Assim, é necessário fazer uma consulta às normas detalhadas.
%
%As referências devem ser apresentadas em ordem alfabética, com as citações no texto obedecendo ao sistema autor-data.Todos os documentos relacionados nas Referências devem ser citados no texto, assim como todas as citações do texto devem constar nas Referências.
%
%\chapter{Exemplo de anexo}
%
%Texto de exemplo, texto de exemplo, texto de exemplo, texto de exemplo, texto de exemplo, texto de exemplo, texto de exemplo, texto de exemplo, texto de exemplo, texto de exemplo, texto de exemplo, texto de exemplo, texto de exemplo, texto de exemplo, texto de exemplo, texto de exemplo, texto de exemplo, texto de exemplo, texto de exemplo.
%
%\section*{1 Exemplo de seção de anexo não apresentada no sumário}
%
%Texto de exemplo, texto de exemplo, texto de exemplo, texto de exemplo, texto de exemplo, texto de exemplo, texto de exemplo, texto de exemplo, texto de exemplo, texto de exemplo, texto de exemplo, texto de exemplo, texto de exemplo, texto de exemplo, texto de exemplo, texto de exemplo, texto de exemplo, texto de exemplo, texto de exemplo.
%
%\subsection*{1.1 Exemplo de subseção de anexo não apresentada no sumário}
%
%Texto de exemplo, texto de exemplo, texto de exemplo, texto de exemplo, texto de exemplo, texto de exemplo, texto de exemplo, texto de exemplo, texto de exemplo, texto de exemplo, texto de exemplo, texto de exemplo, texto de exemplo, texto de exemplo, texto de exemplo, texto de exemplo, texto de exemplo, texto de exemplo, texto de exemplo.
%
%\subsection*{1.2 Exemplo de subseção de anexo não apresentada no sumário}
%
%Texto de exemplo, texto de exemplo, texto de exemplo, texto de exemplo, texto de exemplo, texto de exemplo, texto de exemplo, texto de exemplo, texto de exemplo, texto de exemplo, texto de exemplo, texto de exemplo, texto de exemplo, texto de exemplo, texto de exemplo, texto de exemplo, texto de exemplo, texto de exemplo, texto de exemplo.
%
%\subsection*{1.3 Exemplo de subseção de anexo não apresentada no sumário}
%
%Texto de exemplo, texto de exemplo, texto de exemplo, texto de exemplo, texto de exemplo, texto de exemplo, texto de exemplo, texto de exemplo, texto de exemplo, texto de exemplo, texto de exemplo, texto de exemplo, texto de exemplo, texto de exemplo, texto de exemplo, texto de exemplo, texto de exemplo, texto de exemplo, texto de exemplo.
%
%\section*{2 Exemplo de seção de anexo não apresentada no sumário}
%
%Texto de exemplo, texto de exemplo, texto de exemplo, texto de exemplo, texto de exemplo, texto de exemplo, texto de exemplo, texto de exemplo, texto de exemplo, texto de exemplo, texto de exemplo, texto de exemplo, texto de exemplo, texto de exemplo, texto de exemplo, texto de exemplo, texto de exemplo, texto de exemplo, texto de exemplo.
%
%\section*{3 Exemplo de seção de anexo não apresentada no sumário}
%
%Texto de exemplo, texto de exemplo, texto de exemplo, texto de exemplo, texto de exemplo, texto de exemplo, texto de exemplo, texto de exemplo, texto de exemplo, texto de exemplo, texto de exemplo, texto de exemplo, texto de exemplo, texto de exemplo, texto de exemplo, texto de exemplo, texto de exemplo, texto de exemplo, texto de exemplo.
%
%\chapter{Exemplo de anexo}
%
%Texto de exemplo, texto de exemplo, texto de exemplo, texto de exemplo, texto de exemplo, texto de exemplo, texto de exemplo, texto de exemplo, texto de exemplo, texto de exemplo, texto de exemplo, texto de exemplo, texto de exemplo, texto de exemplo, texto de exemplo, texto de exemplo, texto de exemplo, texto de exemplo, texto de exemplo.
%
%\section*{1 Exemplo de seção de anexo não apresentada no sumário}
%
%Texto de exemplo, texto de exemplo, texto de exemplo, texto de exemplo, texto de exemplo, texto de exemplo, texto de exemplo, texto de exemplo, texto de exemplo, texto de exemplo, texto de exemplo, texto de exemplo, texto de exemplo, texto de exemplo, texto de exemplo, texto de exemplo, texto de exemplo, texto de exemplo, texto de exemplo.
%
%\subsection*{1.1 Exemplo de subseção de anexo não apresentada no sumário}
%
%Texto de exemplo, texto de exemplo, texto de exemplo, texto de exemplo, texto de exemplo, texto de exemplo, texto de exemplo, texto de exemplo, texto de exemplo, texto de exemplo, texto de exemplo, texto de exemplo, texto de exemplo, texto de exemplo, texto de exemplo, texto de exemplo, texto de exemplo, texto de exemplo, texto de exemplo.
%
%\subsection*{1.2 Exemplo de subseção de anexo não apresentada no sumário}
%
%Texto de exemplo, texto de exemplo, texto de exemplo, texto de exemplo, texto de exemplo, texto de exemplo, texto de exemplo, texto de exemplo, texto de exemplo, texto de exemplo, texto de exemplo, texto de exemplo, texto de exemplo, texto de exemplo, texto de exemplo, texto de exemplo, texto de exemplo, texto de exemplo, texto de exemplo.
%
%\subsection*{1.3 Exemplo de subseção de anexo não apresentada no sumário}
%
%Texto de exemplo, texto de exemplo, texto de exemplo, texto de exemplo, texto de exemplo, texto de exemplo, texto de exemplo, texto de exemplo, texto de exemplo, texto de exemplo, texto de exemplo, texto de exemplo, texto de exemplo, texto de exemplo, texto de exemplo, texto de exemplo, texto de exemplo, texto de exemplo, texto de exemplo.
%
%\section*{2 Exemplo de seção de anexo não apresentada no sumário}
%
%Texto de exemplo, texto de exemplo, texto de exemplo, texto de exemplo, texto de exemplo, texto de exemplo, texto de exemplo, texto de exemplo, texto de exemplo, texto de exemplo, texto de exemplo, texto de exemplo, texto de exemplo, texto de exemplo, texto de exemplo, texto de exemplo, texto de exemplo, texto de exemplo, texto de exemplo.
%
%\section*{3 Exemplo de seção de anexo não apresentada no sumário}
%
%Texto de exemplo, texto de exemplo, texto de exemplo, texto de exemplo, texto de exemplo, texto de exemplo, texto de exemplo, texto de exemplo, texto de exemplo, texto de exemplo, texto de exemplo, texto de exemplo, texto de exemplo, texto de exemplo, texto de exemplo, texto de exemplo, texto de exemplo, texto de exemplo, texto de exemplo.
%
%\end{anexosenv}
%
%%---------------------------------------------------------------------
%% INDICE REMISSIVO
%%---------------------------------------------------------------------
%%%%%%MF\phantompart
%%%%%%MF\printindex
%%---------------------------------------------------------------------

\end{document}
