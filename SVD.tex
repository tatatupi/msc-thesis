\appendix\chapter{Decomposi��o em Valores Singulares}

O teste de Grubbs � usado para detec��o de \textit{outliers} em
conjuntos de dados monovariados. Em �ltima an�lise, ele � um teste
de hip�teses convencional que se baseia no fato de que o conjunto de
dados a ser analisado seja razoavelmente aproximado por uma
distribui��o normal.

O teste de Grubbs � capaz de detectar um \textit{outlier} por vez e
o retirar. O teste � realizado iteradas vezes at� que n�o se
encontrem mais \textit{outliers} no conjunto de dados em an�lise.

Seja a seguinte hip�tese nula $H_0$: \lq N�o h� outliers no conjunto
de dados\rq.

Como em todo teste de hip�tese, � necess�rio escolher uma
estat�stica. No caso, define-se a estat�stica do teste de Grubbs
como:

\begin{equation}
G=\frac{max|Y_i|-\bar{Y}}{s}
\end{equation}
em que $\bar{Y}$ � a m�dia amostral e $s$ o desvio padr�o amostral.
$G$ � o m�ximo desvio absoluto em rela��o � m�dia, e � dado em
unidades de desvio padr�o amostral.

Estabelecem-se, ent�o, limites de confian�a de tal forma que, se o
valor de $G$ estiver fora desses limites, o teste pode ser rejeitado
e o ponto $Y_i$ � considerado um \textit{outlier}. Esses limites
podem ser de $\bar{Y}\pm3s$, por exemplo, uma vez que a
probabilidade de uma realiza��o da distribui��o normal estar
compreendida nessa faixa � de $99,7\%$.
