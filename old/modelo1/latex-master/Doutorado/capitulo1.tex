\chapter{Introdu��o}
\label{cap1} \vspace{-1cm}

%
%\begin{flushright}
%\begin{minipage}{0.7\linewidth}
%\emph{``Se queres conversar comigo, define primeiro os termos que usas.''}
%\end{minipage}
%\end{flushright}
%
%\begin{flushright}
%{Voltaire (1694-1778)}
%\end{flushright}
%
%
%
%
%Este cap�tulo consiste na exposi��o preliminar do problema de previs�o de consumo de energia a longo prazo, cuja pesquisa e an�lise foram assumidas por uma equipe da Universidade Federal de Minas Gerais (UFMG), constitu�da de alunos de gradua��o, de p�s-gradua��o e de professores, conforme solicita��o da Companhia Energ�tica de Minas Gerais (CEMIG). O projeto intitula-se \lq\lq Pesquisa de Modelo de Previs�o de Consumo de Energia, para Curto e M�dio Prazo, utilizando Intelig�ncia Computacional\rq\rq, dentro do qual est� inserido o escopo dessa disserta��o de mestrado, que tratar�, especificamente, da detec��o de caracter�sticas nas s�ries temporais de consumo de energia (por exemplo, determinismo) a partir do m�todo de an�lise de dados sub-rogados, e da proposi��o de metodologias de previs�o de consumo de energia a longo prazo. A seguir, define-se o problema e s�o apresentados a motiva��o e os objetivos deste trabalho.
%
%
%\section{Defini��o do Problema}
%
%
%\section{Motiva��o}
%\markright{\thesection ~~~ Motiva��o}
%%\label{motiva}
%
%
%\section{Objetivos do Projeto} \markright{\thesection
%~~~ Objetivos}
%%\label{objetivos}
%
%
%\section{Organiza��o do Trabalho} \markright{\thesection
%~~~ Organiza��o do Trabalho}
%%\label{organiza}
%
%
%\clearpage
