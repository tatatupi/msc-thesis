\chapter*{Ep�grafe}


%\begin{flushright}
%\begin{minipage}{0.7\linewidth}
%\emph{``Fui �tomo, vibrando entre as for�as do Espa�o,\\Devorando amplid�es, em longa e ansiosa espera...\\Part�cula, pousei... Encarcerado, eu era\\Infus�rio do mar em mont�es de sarga�o.\\\\Por s�culos fui planta em movimento escasso,\\Sofri no inverno rude e amei na primavera;\\Depois, fui animal, e no instinto da fera\\Achei a intelig�ncia e avancei passo a passo...\\\\Guardei por muito tempo a express�o dos gorilas,\\Pondo mais f� nas m�os e mais luz nas pupilas,\\A lutar e chorar para, ent�o, compreend�-las!...\\\\Agora, homem que sou, pelo Foro Divino,\\Vivo de corpo em corpo a forjar o destino\\Que me leve a transpor o clar�o das estrelas!...''\\}
%\end{minipage}
%\end{flushright}
%%\vspace{3cm}
%\begin{flushright}
%{Soneto intitulado {\it Jornada}, de autoria de Adelino da Fontoura Chaves. Extra�do do livro {\it Antologia dos Imortais}, psicografado por Francisco C�ndido Xavier.}
%\end{flushright}

%"EVOLU��O3
%
%                                                                                                              Rubens C. %Romanelli

%                De muito longe venho, em surtos milen�rios;

 %               Vivi na luz dos s�is, vaguei por mil esferas

  %              E, preso ao turbilh�o dos motos planet�rios,

   %             Fui lodo e fui cristal, no alvor de priscas eras.

 

    %                            Mil formas animei, nos reinos multif�rios:

     %                           Fui planta no verdor de frescas primaveras

      %                          E, ap�s sombrio est�gio entre os protozo�rios,

       %                         Galguei novos degraus: fui fera dentre as feras.

 

        %        Depois que em mim brilhou o facho da raz�o,

         %       Fui o �ncola feroz das tribos primitivas

          %      E como tal vivi, por vidas sucessivas.

 

           %                     E sempre na espiral da eterna evolu��o,
%
 %                               Um dia eu transporei os c�rculos do mal

  %                              E brilharei na luz da Ess�ncia Universal."

 % O Primado do Esp�rito