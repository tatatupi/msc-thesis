\chapter*{Agradecimentos}

Antes de tudo, sou grato � UFMG por ter me proporcionado toda uma estrutura de excel�ncia, para a realiza��o dos meus estudos. � atrav�s de institui��es p�blicas de ensino superior como ela que podemos evoluir como sociedade. Estendo meus agradecimentos ao PPGEE-UFMG e aos seus funcion�rios, professores e alunos que contribuiram para o desenvolvimento deste trabalho. E ao CNPq, pelo apoio financeiro durante grande parte do meu mestrado.

Agrade�o aos meus orientadores, Prof. Bruno Teixeira e Prof. Leonardo T�rres, por todo o ensinamento e pela forma contagiante com que compartilham suas experi�ncias. Serei eternamente grato por terem me acolhido de volta � Academia, ap�s longos anos.

Aos amigos do Laborat�rio de Modelagem, An�lise e Controle de Sistemas N�o-Lineares (MACSIN), pelo companheirismo e apoio. Aprender junto com voc�s foi incr�vel.

Tamb�m tenho muito a agradecer � equipe do Laborat�rio de Nanoespectroscopia (LabNS), � qual me juntei recentemente, e que tem renovado de forma constante a minha paix�o pela ci�ncia.

�s amigas e aos amigos do Bloco Pr�-Sal, por trazerem leveza e alegria � minha rotina, al�m de serem modelos exemplares que tenho para a vida. � turma de gradua��o da El�trica, em especial ao Igor Baratta, grande amigo e conselheiro. 

Termino esses agradecimentos, mencionando aqueles sem os quais nada faria sentido nem seria poss�vel. Aos meus pais, Marilu e Ubiratan, � minha irm� e ao meu cunhado, Moara e Matheus, agrade�o imensamente pela forma��o, pela confian�a, pelos conselhos e pelo amor incondicional. Voc�s me fazem muito feliz. Aos meus sobrinhos, Iara e Cau�, pelo carinho e pela inspira��o. � minha amada esposa Marina, meu porto seguro. Voc� me faz ser uma pessoa melhor todo dia. E �s minhas fam�lias Melo e Tupinamb�s, pelo amor e pelo incentivo constantes.
\\
