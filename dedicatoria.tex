\chapter*{Dedicat�ria}

\vspace{5cm}
\begin{flushright}
\begin{minipage}{0.7\linewidth}
\emph{A minha querida esposa Anny, que com sua dedica��o e seu amor incondicional me faz sorrir, mesmo nas horas mais dif�ceis.}\\
\end{minipage}
\end{flushright}
