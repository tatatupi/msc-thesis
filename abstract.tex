\chapter*{Abstract}
\addcontentsline{toc}{chapter}{Abstract}

\vspace{-2cm} 
The use of multiple sensors to improve data quality has grown continuously over the last few decades. With the never-ending advances in technology of microprocessors and communication devices, sensor networks will continue to increase in both size and complexity. The most popular applications for fusing data from various sources are related to estimating the states of a dynamic system. For that, two noisy sources of information are needed: a process model, that describes how the states evolve in time; and an observation model, whose data are usually obtained from sensors. Since most sensors are digital, signals must be sampled in order to be processed, leading to sampled-data systems. Classical state estimators in these cases, like the well-known Kalman filter, implicitly consider regularly sampled signals with constant time intervals between samples, such that continuous-time systems can be time discretized into time-invariant representations in most cases. However, because of the widespread use of complex sensor networks without explicit time synchronization, many applications cannot rely on data being transmitted regularly. There are adaptations to state estimation techniques that handle most of the irregularities, provided that timestamps are part of measurement packets and that the increase in computational processing time is acceptable. If timestamps cannot be used in the estimation process, one can either invest in synchronization or accept the assimilation of information at incorrect time instants. The effects in estimation performance of the latter approach has not yet been extensively studied. In this work we investigate how performance is deteriorated by neglecting measurements timestamps in state estimation algorithms. We consider the Poisson process as a model to generate the irregular time instants sequences, and we assess state estimation results for linear and nonlinear systems simulated with aperiodic sampling, using the Kalman filter for the former and its adapted unscented version for the latter. Algorithms are designed to use timestamps or to neglect them in the estimation process, and their results over multiple runs are compared for different simulation scenarios. Finally, we identify and discuss relations between different sets of parameters, such as signal-to-noise ratios and average sampling frequencies, and the degradation in performance.
\\ \\
\textbf{Keywords:} Sensor Fusion; Irregular Sampling; State Estimation; Sampled-data Systems; Time Synchronization; Time-Stamp.
