\chapter*{Abstract}
\addcontentsline{toc}{chapter}{Abstract}

\vspace{-2cm} 
The use of multiple sensors to improve data quality has grown continuously over the last few decades. With the never-ending advances in technology of microprocessors and communication devices, sensor networks will continue to increase in both size and complexity. The most popular applications for fusing data from various sources are related to estimating the states of a dynamic system. For that, two noisy sources of information are needed: a process model, that describes how the states evolve in time; and an observation model, whose data are usually derived from sensors. Since most sensors are digital, signals must be sampled in order to be processed, leading to sampled-data systems. Classical state estimators for sampled-data systems, like the well-known Kalman filter, consider the sampling process as the output of an ideal sampler. In other words, the source of information transmits data at regularly spaced time intervals. Under such consideration, continuous systems can be discretized into time-invariant representations in most cases. However, because of the widespread use of complex sensor networks without time synchronization, many applications cannot rely on data being transmitted regularly. There are adaptations to state estimation techniques that handle most of the irregularities, providing that timestamps are part of the measurement packet and that the increase in computational costs are acceptable. If timestamps cannot be used in the estimation process, one can either invest in synchronization or assimilate the information at incorrect time instants. The effects in estimation performance of the latter approach has not yet been extensively studied. Thus our work investigates how performance is deteriorated by neglecting measurements timestamps in state estimation algorithms. We consider the Poisson process as a model to generate the irregular time instants sequence, and study the errors introduced by shifting the signal samples in time. Then we simulate state estimation results for a linear and a nonlinear system with aperiodic sampling, using Kalman filter for the former and its unscented variation for the latter. Algorithms are designed to use timestamps in the estimation process and to neglect them, and their results over multiple runs are compared for different simulation scenarios. We observe situations where performance of both algorithms differ substantially and others where there are no statistically significant difference. 
\\ \\
\textbf{Keywords:} Sensor Fusion; Irregular Sampling; State Estimation; Sampled-data Systems; Time Synchronization; Time-Stamp.

