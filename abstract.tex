\chapter*{Abstract}
\addcontentsline{toc}{chapter}{Abstract}

\vspace{-2cm}Over the last two decades, subspace identification
methods have attracted great attention due their potential for
application in industry, especially for multivariable systems. The
algorithms of subspace identification are as easy to implement as
well-known algorithms such as least squares. However, the theory
behind these methods requires concepts from linear systems,
stochastic processes, system identification, linear algebra, and
others, making their understanding more difficult. As a consequence
such methods are less well-known than others. This work investigates
the use of subspace identification techniques applied to
discrete-time, linear, time-invariant and multivariable systems. Our
efforts have focused on the following objectives: (I) to
geometrically interpret the methods, (II) to investigate, by means
of simulations, situations in which the algorithms are best
indicated, (III) to compare with other classical methods and (IV) to
apply them to simulated and experimental systems. At first, basic
concepts are reviewed: data modeling in state space, block matrices
and vectors, geometric and statistical tools. These concepts are
critical to understanding the theory behind subspace identification.
Later, in the case of deterministic subspace identification, a
comprehensive study about how state matrices can be obtained from
input-output data is provided. In this study, the algorithms N4SID
and MOESP are presented. Subsequently, the stochastic case is
treated similarly to the deterministic one. It is shown that the
methods N4SID and MOESP are robust to white measurement noise.
However, when the methods are exposed to colored noise, either
measurement or process, these estimators are biased. One alternative
is to use instrumental variable methods. Two such algorithms are
presented: MOESP-PI and MOESP-PO. Demonstrations and simulated
examples are presented, in order to facilitate the understanding of
the characteristics of subspaces methods. Finally, the methods
N4SID, MOESP, MOESP-PO, MOESP-PI are applied to three multivariable
systems. The first system is a simulated model of a DC motor. The
other two are experimental systems, namely water pumping system and
a column flotation system. The results suggest that the subspace
methods are a feasible alternative for linear systems of multiple
inputs and multiple outputs.
\\

\textbf{Keywords:} System identification; Subspace methods;
State-space models; Multivariable systems; Linear systems.
