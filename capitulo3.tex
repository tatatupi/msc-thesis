%------------------------------------------------------------------------------
\chapter{Sensor Fusion}
\vspace{-1cm} \label{cap2}

%\begin{flushright}
%\begin{minipage}{0.7\linewidth}
%\emph{``Quando uma criatura humana desperta para um grande sonho e
%sobre ele lan�a toda a for�a de sua alma, todo o universo conspira a
%seu favor.''}
%\end{minipage}
%\end{flushright}
%
%\begin{flushright}
%{Goethe}
%\end{flushright}

%\vspace{1cm}

%%Colocar uma descri��o do cap�tulo aqui!
%\section{Introdu��o}\label{sec int_cap_2}


In this chapter, we review sensor fusion techniques. We start with a brief explanation on the four different types: possibilistic, fuzzy reasoning, evidential belief e probabilistic. The latter is further explained, since it is the one used in the simulated examples. Specific techniques to deal with irregular sampling in sensor fusion are also addressed.

\section{Introduction}

Explicar brevemente sobre os tipos de fus�o sensorial: possibilistic, fuzzy reasoning, evidential belief e probabilistic \citep{Khaleghi2013}. Focar no probabilistic. 


\section{Probabilistic Sensor Fusion}

Entrar em maiores detalhes do probabilistic, explicando KF e o PF (caso seja utilizado no trabalho).

\subsection{Kalman Filters}

\subsection{Particle Filters}

\section{Sensor Fusion Techniques}

Para fus�o de informa��es m�ltiplas de observa��es, explicar os quatro grandes grupos: parallel filter (multi-output system, multi-output KF), sequential filter (kfusing first output as the prediction for the secod output), outputs fusion (outputs are fused using the noise covariance) e track-to-track fusion (single-output KFs fused considering correlation). \citep{Willner1976,Fatehi2017}

Tentar agrupar os principais m�todos de filtragem, por tipo de irregularidade (possivelmente uma grande tabela)


\clearpage