%------------------------------------------------------------------------------
\chapter{Sensor Fusion}
\vspace{-1cm} \label{cap2}

%\begin{flushright}
%\begin{minipage}{0.7\linewidth}
%\emph{``Quando uma criatura humana desperta para um grande sonho e
%sobre ele lan�a toda a for�a de sua alma, todo o universo conspira a
%seu favor.''}
%\end{minipage}
%\end{flushright}
%
%\begin{flushright}
%{Goethe}
%\end{flushright}

%\vspace{1cm}

%%Colocar uma descri��o do cap�tulo aqui!
%\section{Introdu��o}\label{sec int_cap_2}


In this chapter, we review sensor fusion techniques. We start with a brief explanation on the four different types: possibilistic, fuzzy reasoning, evidential belief e probabilistic. The latter is further explained, since it is the one used in the simulated examples. Specific techniques to deal with irregular sampling in sensor fusion are also addressed.

\section{Introduction}

The idea that combining information from multiple sensors to improve overall system performances has been in discussion for several decades. In the early days, there were those who argued against the synergism hype that was being spread in military systems, using the multisensor concept \citep{Fowler1979}. In this work, Fowler created what he called his seventh law: 

\begin{quotation}
	"Be wary of proposals for synergistic systems. Most of the time when you try to make 2 + 2 = 5, you end up with 3... and sometimes 1.9"
\end{quotation}

Although he was right to affirm that the added complexity and high costs were not always worth it, many posterior studies advocated that the fusion of sensor data will always be better, in the sense that the probability of correctly classifying a target increases. A direct answer to Fowler's work came in the year after \citep{Nahin1980}, where Nahin and Pokoski used strict concepts and definitions, but also acknowledged the assumptions of discarding network complexity and costs for the sake of their arguments. This topic continued to draw scientific attention throughout the years, like the work of \citep{Rao1998} and \citep{Dasarathy2000}. Rao focused on fusion methods and its comparison to classifiers' performance and establishing conditions that guarantee that the fused system will at least perform as good as the best classifier. Dasarathy's work extends that of Rao's, but was able to show a scenario at which a two-sensor suite outperforms a three sensor suite from a parametric fusion benefits domain perspective. 

\citep{Theil2000} created performance metrics for sensor fusion.

Despite all philosophical discussions, the field of study have been investigated and applied in many real applications, like remote sensing \citep{Foster1981}, robotics \citep{Richardson1988} and intelligent systems \citep{Luo1989}. Recently, with the modernization and popularization of sensors, it has grown significantly, with hot topics emerging in the area, such as body sensor networks for health-care applications \citep{Gravina2017} or artificial intelligence \citep{Safari2014, Jordao2018}.

Recent reviews of the state of the art provide a very broad understanding of the area and its advances \citep{Khaleghi2013, Jing2013}.


\subsection{Motivation and Advantages}

elmenreich 
infofusion

\subsection{Taxonomy and Classification}

Classifica��o: bolstrom
3-level fusion (jing progress in china)
Introduction to multisensor (Hall)
Input / Output categorization (dasarathy and elmenreich)

probabilistic, fuzzy reasoning, evidential belief, possibilistic, rought set \citep{Khaleghi2013}. 

\section{Probabilistic Sensor Fusion}

Entrar em maiores detalhes do probabilistic, explicando KF e o PF (caso seja utilizado no trabalho).

\subsection{Kalman Filters}

\subsection{Particle Filters}

\section{Sensor Fusion Techniques}

Para fus�o de informa��es m�ltiplas de observa��es, explicar os quatro grandes grupos: parallel filter (multi-output system, multi-output KF), sequential filter (kfusing first output as the prediction for the secod output), outputs fusion (outputs are fused using the noise covariance) e track-to-track fusion (single-output KFs fused considering correlation). \citep{Willner1976,Fatehi2017}

Tentar agrupar os principais m�todos de filtragem, por tipo de irregularidade (possivelmente uma grande tabela)


\clearpage